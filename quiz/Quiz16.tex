\documentclass{beamer}

\usepackage{graphicx}
\usepackage[utf8]{inputenc}
\usepackage[T1]{fontenc}
\usepackage[english]{babel}
\usepackage{listings}
\usepackage{xcolor}
\usepackage{eso-pic}
\usepackage{amsmath}
\usepackage{mathrsfs}
\usepackage{url}
\usepackage{amssymb}
\usepackage{amsmath}
\usepackage{multirow}
\usepackage{hyperref}
\usepackage{booktabs}
\usepackage{bbm}
\usepackage{cooltooltips}
\usepackage{colordef}
\usepackage{beamerdefs}
\usepackage{lvblisting}

\pgfdeclareimage[height=3.5cm]{logobig}{hucaselogo.pdf}
\pgfdeclareimage[height=0.7cm]{logosmall}{LOB_Logo.pdf}

\renewcommand{\titlescale}{1.0}
\renewcommand{\titlescale}{1.0}
\renewcommand{\leftcol}{0.6}

\title[Selected topics in Mathematical Statistics, Quiz 16]{Selected topics in Mathematical Statistics, Quiz 16}
\authora{Nina Lesnik}
\authorb{}
\authorc{}

\def\linka{http://lvb.wiwi.hu-berlin.de}
\def\linkb{http://case.hu-berlin.de}
\def\linkc{}

\institute{Ladislaus von Bortkiewicz Chair of Statistics \\
C.A.S.E. -- Center for Applied Statistics\\
and Economics\\
Humboldt--Universität zu Berlin \\}

\hypersetup{pdfpagemode=FullScreen}

\begin{document}

% 0-1
%%%%%%%%%%%%%%%%%%%%%%%%%%%%%%%%%%%%%%%%
\frame[plain]{

\titlepage
}

%%%%%%%%%%%%%%%%%%%%%%%%%%%%%%%%%%%%%%%%
\section{Motivation}
\frame{
\frametitle{Motivation}
\begin{itemize}
\item  i.i.d random variables - with CLT we can approximate $\overline{X}=\frac{\sum_{i=1}^{n}X_i}{n} \sim N(\mu, \frac{\sigma^2}{n})$
\item we would like to increase our approximation
\item therefore we need some additional information 
\item \textbf{Edgeworth approximation} - use information about higher order moments to increase the accuracy
\item especially important  if we have small sample
\end{itemize}


}



%%%%%%%%%%%%%%%%%%%%%%%%%%%%%%%%%%%%%%%%

\frame{
\frametitle{Edgeworth Expansion}

Edgeworth Expansion can be applied in case $3$rd and $4$th moment are known.\\

$$G_n(x) \sim \phi(x)- \frac{\beta_1(x^2-1)}{6n^(1/2)}\varphi(x) - \big\{\frac{\beta_2(x^3-3x)}{24n}+ $$$$ \frac{\beta_1^2(x^5-10x^3 + 15x)}{72n}\big\}\varphi(x) $$

Where $\varphi$ and $\phi$ are pdf and cdf of standard normal distribution. $\beta_1$ is standardized 3rd moment and $\beta_2$ is the excess kurtosis, $\frac{\mu_4}{\sigma^4}-3$
}

%%%%%%%%%%%%%%%%%%%%%%%%%%%%%%%%%%%%%%%%
\section{Quiz 16}
\frame{
\frametitle{Quiz 16}
\textbf{Example}
A sample of 5 values from the standard exponential distribution gives us:
$$\mu=1,  \sigma^2=1,  \beta_1=2,  \beta_2=6$$
Prove the result


}

%%%%%%%%%%%%%%%%%%%%%%%%%%%%%%%%%%%%%%%%

\frame{


standard exponential distribution: $ f(x) = e^{-x}$ \\
\textbf{Calculate $\mu$}:

\begin{flushleft}

$\mu = E[x]= \int_{0}^{\infty}x e^{-x}dx = - xe^{-x} \Big|_{0}^{\infty} + \int_{0}^{\infty} e^{-x}dx =1$ 
\end{flushleft}
\textbf{Calculate $\sigma^2$:}\\
$$\sigma^2 = E[x^2] - E[x]^2 $$
\begin{flushleft}
$ E[x^2] =  \int_{0}^{\infty}x ^2 e^{-x}dx =   - x^2e^{-x} \Big|_{0}^{\infty} + \int_{0}^{\infty}2x e^{-x}dx = 0-0 + 2\int_{0}^{\infty} x e^{-x}dx=2
$ \end{flushleft}
$$\sigma^2 = 2-1=1$$


}


%%%%%%%%%%%%%%%%%%%%%%%%%%%%%%%%%%%%%%%%

\frame{

\textbf{Calculate $\beta_1$:}

$$\beta_1 = \frac{\mu_3}{\sigma^3}= \frac{E[(X- \mu)^3]}{\sqrt{E[(X-\mu)^2]}^3}$$

\begin{flushleft}
$E[(X- \mu)^3]= \int_{0}^{\infty}(x-1)^3 e^{-x}dx =\int_{0}^{\infty}x^3 e^{-x}dx -\int_{0}^{\infty}3x^2 e^{-x}dx +\int_{0}^{\infty}3x e^{-x}dx - \int_{0}^{\infty} e^{-x}dx  =\int_{0}^{\infty}x^3 e^{-x}dx  -6 + 3 -1= - x^3e^{-x} \Big|_{0}^{\infty} + \int_{0}^{\infty} 3x^2e^{-x}dx -24= 6-4 = 2$\\
\end{flushleft}
\begin{flushleft}
$E[(X- \mu)^2]= \int_{0}^{\infty}(x-1)^2 e^{-x}dx =\int_{0}^{\infty}x^2 e^{-x}dx -\int_{0}^{\infty}2x e^{-x}dx +\int_{0}^{\infty} e^{-x}dx  =2-2+1=1$\\

\end{flushleft}
$$\beta_1 = \frac{2}{\sqrt{1}^3}= 2$$

}


%%%%%%%%%%%%%%%%%%%%%%%%%%%%%%%%%%%%%%%%


\frame{

\textbf{Calculate $\beta_2$:}

$$\beta_2 = \frac{\mu_4}{\sigma^4}-3= \frac{E[(X- \mu)^4]}{\sqrt{E[(X-\mu)^2]}^4}-3$$

\begin{flushleft}
$E[(X- \mu)^4]= \int_{0}^{\infty}(x-1)^4 e^{-x}dx =\int_{0}^{\infty}x^4 e^{-x}dx  -\int_{0}^{\infty}4x^3 e^{-x}dx +\int_{0}^{\infty}6x^2 e^{-x}dx -\int_{0}^{\infty}4x e^{-x}dx + \int_{0}^{\infty} e^{-x}dx  = - x^4e^{-x} \Big|_{0}^{\infty}+\int_{0}^{\infty}4x^3 e^{-x}dx  -24 +12 -4 +1=24-24+9=9$\\
\end{flushleft}

$$\beta_2 = \frac{9}{\sqrt{1}^4}-3= 6$$

}

\end{document}