% Type of the document
\documentclass{beamer}

% elementary packages:
\usepackage{graphicx}
\usepackage[latin1]{inputenc}
\usepackage[T1]{fontenc}
\usepackage[english]{babel}
\usepackage{listings}
\usepackage{xcolor}
\usepackage{eso-pic}
\usepackage{mathrsfs}
\usepackage{url}
\usepackage{amssymb}
\usepackage{amsmath}
\usepackage{multirow}
\usepackage{hyperref}
\usepackage{booktabs}

% additional packages
\usepackage{bbm}

% packages supplied with ise-beamer:
\usepackage{cooltooltips}
\usepackage{colordef}
\usepackage{beamerdefs}
\usepackage{lvblisting}

% Change the pictures here:
% logobig and logosmall are the internal names for the pictures: do not modify them. 
% Pictures must be supplied as JPEG, PNG or, to be preferred, PDF
\pgfdeclareimage[height=2cm]{logobig}{hulogo}
% Supply the correct logo for your class and change the file name to "logo". The logo will appear in the lower
% right corner:
\pgfdeclareimage[height=0.7cm]{logosmall}{Figures/LOB_Logo}

% Title page outline:
% use this number to modify the scaling of the headline on title page
\renewcommand{\titlescale}{1.0}
% the title page has two columns, the following two values determine the percentage each one should get
\renewcommand{\titlescale}{1.0}
\renewcommand{\leftcol}{0.6}

% Define the title.Don't forget to insert an abbreviation instead 
% of "title for footer". It will appear in the lower left corner:
\title[Quiz 19]{Mathematical Foundations\\
 for Finance and Insurance \\
}
% Define the authors:
\authora{Kristian Boroz } % a-c
%\authorb{Stephan Stahlschmidt}
%\authorc{}

% Define any internet addresses, if you want to display them on the title page:
\def\linka{http://lvb.wiwi.hu-berlin.de}
\def\linkb{}
\def\linkc{}
% Define the institute:
\institute{Ladislaus von Bortkiewicz Chair of Statistics \\
Humboldt--Universit{\"a}t zu Berlin \\}

% Comment the following command, if you don't want, that the pdf file starts in full screen mode:
\hypersetup{pdfpagemode=FullScreen}

%Start of the document
\begin{document}

% create the title slide, layout controlled in beamerdefs.sty and the foregoing specifications
\frame[plain]{
\titlepage
}

% The titles of the different sections of you talk, can be included via the \section command. The title will be displayed in the upper left corner. To indicate a new section, repeat the \section command with, of course, another section title
%%%%%%%%%%%%%%%%%%%%%%%%%%%%%%%%%%%%%%%%%%%%%%%%%%%%%%%%%%%%%%%%%%%%%%%%%%%%%%%%%%%%%%%%%%%%%%%%%%%%%%%%%%%%%%%%%%%%%%%%
\section{Introduction}
%%%%%%%%%%%%%%%%%%%%%%%%%%%%%%%%%%%%%%%%%%%%%%%%%%%%%%%%%%%%%%%%%%%%%%%%%%%%%%%%%%%%%%%%%%%%%%%%%%%%%%%%%%%%%%%%%%%%%%%%

% (A numbering of the slides can be useful for corrections, especially if you are
% dealing with large tex-files)
%%%%%%%%%%%%%%%%%%%%%%%%%%%%%%%%%%%%%%%%%%%%%%%%%%%%%%%%%%%%%%%%%%%%%%%%%%%%%%%%%%%%%%%%%%%%%%%%%%%%%%%%%%%%%%%%%%%%%%%%
\frame{
\frametitle{Quiz 19}
Prove Theorem 52 as an application of Theorem 38, instead
of using Lemma 51.
}


%%%%%%%%%%%%%%%%%%%%%%%%%%%%%%%%%%%%%%%%%%%%%%%%%%%%%%%%%%%%%%%%%%%%%%%%%%%%%%%%%%%%%%%%%%%%%%%%%%%%%%%%%%%%%%%%%%%%%%%%
\frame{
\frametitle{Outline}

\begin{enumerate}
\item Motivation \quad \checkmark
\item Theorems
\item Proof
\end{enumerate}
}

\section{Theorems}
%%%%%%%%%%%%%%%%%%%%%%%%%%%%%%%%%%%%%%%%%%%%%%%%%%%%%%%%%%%%%%%%%%%%%%%%%%%%%%%%%%%%%%%%%%%%%%%%%%%%%%%%%%%%%%%%%%%%%%%%
\frame{
\frametitle{Theorem 38}
\begin{theorem}[Dvoretzky, Kiefer, and Wolfowitz]

Let F be defined on $\mathbb{R}$, $\exists C > 0\:(not\:depending\:on\:F)\:such\:that\:$

\end{theorem}
\[
  P(D_{n}>d)<Cexp(-2nd^{2}),\:\:\:\:d>0
\]
$\forall n=1,2,...$
}

%%%%%%%%%%%%%%%%%%%%%%%%%%%%%%%%%%%%%%%%%%%%%%%%%%%%%%%%%%%%%%%%%%%%%%%%%%%%%%%%%%%%%%%%%%%%%%%%%%%%%%%%%%%%%%%%%%%%%%%%

\frame{
\frametitle{Theorem 52}
\begin{theorem}[52]
Let\:$0<p<1$.\:Suppose\:that\:$\xi_{p}$\:is\:the\:unique\:solution\:$x$\:of\\$F(x-)\leq p\leq F(x)$.\:Then\:for\:every\:$\epsilon>0$
\end{theorem}
\[
  P(\:|\:\widehat{{\xi}_{pn}}-\xi_{p}|>\epsilon)\leq2exp(-2n\delta_{\epsilon}^{2}),\forall n
\]
$where\:\delta_{\epsilon}=min(F(\xi_{p}+\epsilon)-p,p-F(\xi_{p}+\epsilon)$
}



%%%%%%%%%%%%%%%%%%%%%%%%%%%%%%%%%%%%%%%%%%%%%%%%%%%%%%%%%%%%%%%%%%%%%%%%%%%%%%%%%%%%%%%%%%%%%%%%%%%%%%%%%%%%%%%%%%%%%%%%


\section{Proof}


%%%%%%%%%%%%%%%%%%%%%%%%%%%%%%%%%%%%%%%%%%%%%%%%%%%%%%%%%%%%%%%%%%%%%%%%%%%%%%%%%%%%%%%%%%%%%%%%%%%%%%%%%%%%%%%%%%%%%%%%
\frame{
\frametitle{Proof}
\begin{theorem}[52]
\[
  P(\:|\:\widehat{{\xi}_{pn}}-\xi_{p}|>\epsilon)\leq2exp(-2n\delta_{\epsilon}^{2})
\]
\end{theorem}
\begin{theorem}[38]
\[
  P(D_{n}>d)<Cexp(-2nd^{2})
\]
\end{theorem}


}

%%%%%%%%%%%%%%%%%%%%%%%%%%%%%%%%%%%%%%%%%%%%%%%%%%%%%%%%%%%%%%%%%%%%%%%%%%%%%%%%%%%%%%%%%%%%%%%%%%%%%%%%%%%%%%%%%%%%%%%%
\frame{
\frametitle{Proof}
\begin{theorem}[52]
\[
  P(\:|\:\widehat{{\xi}_{pn}}-\xi_{p}|>\epsilon)\leq2exp(-2n\delta_{\epsilon}^{2})
\]
\end{theorem}
\begin{theorem}[38]
\[
  P(D_{n}>d)<Cexp(-2nd^{2})
\]
\end{theorem}
$D_{n}=KS-Distance=sup|F_{n}(x)-F(x)|$

}

%%%%%%%%%%%%%%%%%%%%%%%%%%%%%%%%%%%%%%%%%%%%%%%%%%%%%%%%%%%%%%%%%%%%%%%%%%%%%%%%%%%%%%%%%%%%%%%%%%%%%%%%%%%%%%%%%%%%%%%%
\frame{
\frametitle{Proof}
\begin{theorem}[52]
\[
  P(\:|\:\widehat{{\xi}_{pn}}-\xi_{p}|>\epsilon)\leq2exp(-2n\delta_{\epsilon}^{2})
\]
\end{theorem}
\begin{theorem}[38]
\[
  P(sup|F_{n}(x)-F(x)|>d)<Cexp(-2nd^{2})
\]
\end{theorem}


}

%%%%%%%%%%%%%%%%%%%%%%%%%%%%%%%%%%%%%%%%%%%%%%%%%%%%%%%%%%%%%%%%%%%%%%%%%%%%%%%%%%%%%%%%%%%%%%%%%%%%%%%%%%%%%%%%%%%%%%%%
\frame{
\frametitle{Proof}
\begin{theorem}[52]
\[
  P(\:|\:\widehat{{\xi}_{pn}}-\xi_{p}|>\epsilon)\leq2exp(-2n\delta_{\epsilon}^{2})
\]
\end{theorem}
\begin{theorem}[38]
\[
  P(|F_{n}(x)-F(x)|>d)\leq\:Cexp(-2nd^{2})
\]
\end{theorem}


}

%%%%%%%%%%%%%%%%%%%%%%%%%%%%%%%%%%%%%%%%%%%%%%%%%%%%%%%%%%%%%%%%%%%%%%%%%%%%%%%%%%%%%%%%%%%%%%%%%%%%%%%%%%%%%%%%%%%%%%%%
\frame{
\frametitle{Proof}
\begin{theorem}[52]
\[
  P(\:|\:\widehat{{\xi}_{pn}}-\xi_{p}|>\epsilon)\leq2exp(-2n\delta_{\epsilon}^{2})
\]
\end{theorem}
\begin{theorem}[38]
\[
  P(|F_{n}(x)-F(x)|>d)\leq\:Cexp(-2nd^{2})
\]
\end{theorem}
Using the Strong Consistency property and the following definition: 
Given a sample $X_{1},...,X_{n}$ of observations on $F$, the sample $p$th
quantile $\widehat{{\xi}_{pn}}$ or $\widehat{{\xi}_{p}}$, is defined as the $p$th quantile of the sample
distribution function $F_{n}$

}

%%%%%%%%%%%%%%%%%%%%%%%%%%%%%%%%%%%%%%%%%%%%%%%%%%%%%%%%%%%%%%%%%%%%%%%%%%%%%%%%%%%%%%%%%%%%%%%%%%%%%%%%%%%%%%%%%%%%%%%%
\frame{
\frametitle{Proof}
\begin{theorem}[52]
\[
  P(\:|\:\widehat{{\xi}_{pn}}-\xi_{p}|>\epsilon)\leq2exp(-2n\delta_{\epsilon}^{2})
\]
\end{theorem}
\begin{theorem}[38]
\[
  P(\:|\:\widehat{{\xi}_{pn}}-\xi_{p}|>\epsilon)\leq\:Cexp(-2nd^{2})
\]
\end{theorem}


}

%%%%%%%%%%%%%%%%%%%%%%%%%%%%%%%%%%%%%%%%%%%%%%%%%%%%%%%%%%%%%%%%%%%%%%%%%%%%%%%%%%%%%%%%%%%%%%%%%%%%%%%%%%%%%%%%%%%%%%%%
\frame{
\frametitle{Proof}
\begin{theorem}[52]
\[
  P(\:|\:\widehat{{\xi}_{pn}}-\xi_{p}|>\epsilon)\leq2exp(-2n\delta_{\epsilon}^{2})
\]
\end{theorem}
\begin{theorem}[38]
\[
  P(\:|\:\widehat{{\xi}_{pn}}-\xi_{p}|>\epsilon)\leq\:Cexp(-2nd^{2})
\]
\end{theorem}
Using the following notation:
$\delta_{\epsilon}=F(\zeta_{p}+\epsilon)-p\rightarrow\delta_{\epsilon}\sim\epsilon$ 


}


%%%%%%%%%%%%%%%%%%%%%%%%%%%%%%%%%%%%%%%%%%%%%%%%%%%%%%%%%%%%%%%%%%%%%%%%%%%%%%%%%%%%%%%%%%%%%%%%%%%%%%%%%%%%%%%%%%%%%%%%
\frame{
\frametitle{Proof}
\begin{theorem}[52]
\[
  P(\:|\:\widehat{{\xi}_{pn}}-\xi_{p}|>\epsilon)\leq\:2exp(-2n\delta_{\epsilon}^{2})
\]
\end{theorem}
\begin{theorem}[38]
\[
  P(\:|\:\widehat{{\xi}_{pn}}-\xi_{p}|>\epsilon)\leq\:Cexp(-2n\delta_{\epsilon}^{2})
\]
\end{theorem}


}

%%%%%%%%%%%%%%%%%%%%%%%%%%%%%%%%%%%%%%%%%%%%%%%%%%%%%%%%%%%%%%%%%%%%%%%%%%%%%%%%%%%%%%%%%%%%%%%%%%%%%%%%%%%%%%%%%%%%%%%%
\frame{
\frametitle{Proof}

For any integer $n$ and any $\lambda$ not less than $\sqrt{log(2)/2}$
and $\gamma n^{-1/6}$where $\gamma=1.0841$, we have:
\[
 P(D_{n}>n)\leq exp(-2\lambda^{2})
\]

\begin{proof}


First Step:
\\where $n\geq39$ and $\lambda\leq\frac{\sqrt{n}}{2}$:
$C=3.61$\\
\\
Second Step:
\\ where $n\leq38$ and $\lambda>\frac{\sqrt{n}}{2}$:
$C=2$
\end{proof}

}

%%%%%%%%%%%%%%%%%%%%%%%%%%%%%%%%%%%%%%%%%%%%%%%%%%%%%%%%%%%%%%%%%%%%%%%%%%%%%%%%%%%%%%%%%%%%%%%%%%%%%%%%%%%%%%%%%%%%%%%%
\frame{
\frametitle{Proof}
\begin{theorem}[52]
\[
  P(\:|\:\widehat{{\xi}_{pn}}-\xi_{p}|>\epsilon)\leq\:2exp(-2n\delta_{\epsilon}^{2})
\]
\end{theorem}
\begin{theorem}[38]
\[
  P(\:|\:\widehat{{\xi}_{pn}}-\xi_{p}|>\epsilon)\leq\:2exp(-2n\delta_{\epsilon}^{2})
\]
\end{theorem}


}

%%%%%%%%%%%%%%%%%%%%%%%%%%%%%%%%%%%%%%%%%%%%%%%%%%%%%%%%%%%%%%%%%%%%%%%%%%%%%%%%%%%%%%%%%%%%%%%%%%%%%%%%%%%%%%%%%%%%%%%%




\section{Further Information}
%%%%%%%%%%%%%%%%%%%%%%%%%%%%%%%%%%%%%%%%%%%%%%%%%%%%%%%%%%%%%%%%%%%%%%%%%%%%%%%%%%%%%%%%%%%%%%%%%%%%%%%%%%%%%%%%%%%%%%%%

%%%%%%%%%%%%%%%%%%%%%%%%%%%%%%%%%%%%%%%%%%%%%%%%%%%%%%%%%%%%%%%%%%%%%%%%%%%%%%%%%%%%%%%%%%%%%%%%%%%%%%%%%%%%%%%%%%%%%%%%
\frame{
\frametitle{Further Information}


\begin{thebibliography}{aaaaaaaaaaaaaaaaa}
\beamertemplatearticlebibitems
\bibitem{Oetiker:2006}
Massart 
\newblock{\em The
tight constant in the Dvoretzky-Kiefer-Wolfowitz inequality}
\newblock The Annals of Probability, Vol 18, No.3, 1269-1283 (1990)
\beamertemplatebookbibitems
\bibitem{Eckel:2004}
Dudley
\newblock {\em Uniform
Central Limit Theorems}
\newblock Cambridge
University Press, 2. Ed. (2014)
\end{thebibliography}
}



% Define the end of the document:
\end{document}
