\documentclass{beamer}

\usepackage{graphicx}
\usepackage[utf8]{inputenc}
\usepackage[T1]{fontenc}
\usepackage[english]{babel}
\usepackage{listings}
\usepackage{xcolor}
\usepackage{eso-pic}
\usepackage{amsmath}
\usepackage{mathrsfs}
\usepackage{url}
\usepackage{amssymb}
\usepackage{amsmath}
\usepackage{multirow}
\usepackage{hyperref}
\usepackage{booktabs}
\usepackage{bbm}
\usepackage{cooltooltips}
\usepackage{colordef}
\usepackage{beamerdefs}
\usepackage{lvblisting}

\pgfdeclareimage[height=3.5cm]{logobig}{hucaselogo.pdf}
\pgfdeclareimage[height=0.7cm]{logosmall}{LOB_Logo.pdf}

\renewcommand{\titlescale}{1.0}
\renewcommand{\titlescale}{1.0}
\renewcommand{\leftcol}{0.6}

\title[MSM: Quiz 15a]{Selected Topics in Mathematical Statistics, Quiz 15a}
\authora{Alma Osmic}
\authorb{}
\authorc{}

\def\linka{http://lvb.wiwi.hu-berlin.de}
\def\linkb{http://case.hu-berlin.de}
\def\linkc{}

\institute{Ladislaus von Bortkiewicz Chair of Statistics \\
Humboldt--Universit{\"a}t zu Berlin \\}

\hypersetup{pdfpagemode=FullScreen}

\begin{document}

% 0-1
%%%%%%%%%%%%%%%%%%%%%%%%%%%%%%%%%%%%%%%%
\frame[plain]{

\titlepage

}

%%%%%%%%%%%%%%%%%%%%%%%%%%%%%%%%%%%%%%%%
\section{Quiz 15a}
\frame{
\frametitle{Quiz 15a}
\textbf{Properties of $M_X$:}
\vfill
\begin{itemize}
 \item $M_{X}(0) = 1$
 \item $M_{aX}(t) = M_{X}(at), \ \forall a \in \mathbb{R}$
 \item $EX^{n} = \frac{d^n}{dt^n}M_{X}(0)$
 \item Let $X_{1},..., X_{n}$ iid, $S_n = \sum_{i = 1}^{n} X_i$ then $M_{S_n}(t) = \prod_{i=1}^{n} M_{X_i}(t)$
\end{itemize}

\vfill
Quiz 15a: Verify the properties of $M_X$
}

%%%%%%%%%%%%%%%%%%%%%%%%%%%%%%%%%%%%%%%%
\frame{
\frametitle{Quiz 15a}

The Moment Generating Function is defined as:

$$M_{X}(t) = E[e^{t^{T}X}] = \int ... \int exp(t^{T}X)dF_{X}, \ t \in \mathbb{R}^k$$

The idea behind is, that
$$e^{tX} = 1 + tX + \frac{t^{2}X^{2}}{2!} + \frac{t^{3}X^{3}}{3!} + \cdots,$$

Thus

$$M_{X}(t) = E[e^{tX}] = 1 + tm_1 + \frac{t^{2}m_{2}}{2!} + \frac{t^{3}m_{3}}{3!} + \cdots,$$

where $m_i$ is the $i$th moment.

}


%%%%%%%%%%%%%%%%%%%%%%%%%%%%%%%%%%%%%%%%

\frame{
\frametitle{Quiz 15a}


First and second property:
\vfill
\begin{itemize}
 \item $\mathbf{M_{X}(0) = 1}$:
	$$M_{X}(0) = \mathbbm{E}[e^{0X}] = \mathbbm{E}[1] = 1$$
 
 \item $\mathbf{M_{aX}(t) = M_{X}(at), \ \forall a \in \mathbb{R}}$ 
	$$M_{aX} = \mathbbm{E}[e^{atX}] = M_{X}(at)$ for any $a \in \mathbb{R}$$

\end{itemize}

}


%%%%%%%%%%%%%%%%%%%%%%%%%%%%%%%%%%%%%%%%

\frame{
\frametitle{Quiz 15a}


Third property:
\vfill
\begin{itemize}
 \item $\mathbf{EX^{n} = \frac{d^n}{dt^n}M_{X}(0)}$:
\end{itemize}

\begin{flushleft}
\begin{aligned}
 \frac{d^n}{dt^n}M_{X}(t) &= \frac{d^n}{dt^n}E[exp(tX)] = \\
&= E \left[\frac{d^n}{dt^n} exp(tX) \right] = E[X^{n} exp(tX)]
\end{aligned}
\end{flushleft}

calculate for $t = 0$ and we get:

$$E[X^{n}exp(0X)] = E[X^n]$$

}

}


%%%%%%%%%%%%%%%%%%%%%%%%%%%%%%%%%%%%%%%%

\frame{
\frametitle{Quiz 15a}


Fourth property:
\vfill
\begin{itemize}
 \item \textbf{Let $X_{1},..., X_{n}$ iid, $S_n = \sum_{i = 1}^{n} X_i$ then $M_{S_n}(t) = \prod_{i=1}^{n} M_{X_i}(t)$}
\end{itemize}

\begin{flushleft}
\begin{aligned}
M_{S_n}(t) &= E[exp(tS_n)] = E \left[exp(t\sum_{i=1}^{n}X_{i}) \right] = E \left[exp(\sum_{i=1}^{n}tX_{i}) \right] \\
				 &= E \left[\prod_{i=1}^{n} exp(tX_{i}) \right] = \prod_{i=1}^{n} E[exp(tX_{i})] = \prod_{i=1}^{n} M_{X_i}(t)
\end{aligned}
\end{flushleft}

Last two steps were possible because of the mutual independence of variables $X_i$.


}





\end{document}